\documentclass{article}

\usepackage[utf8x]{inputenc}
\usepackage[english,russian]{babel}
\usepackage{cmap}
\usepackage{commath}
\usepackage{amsmath}
\usepackage{amsfonts}
\usepackage{mathtools}
\usepackage{amssymb}
\usepackage{parskip}
\usepackage{titling}
\usepackage{color}
\usepackage{hyperref}
\usepackage{cancel}
\usepackage{enumerate}
\usepackage{multicol}
\usepackage{graphicx}
\usepackage{docmute}
\usepackage[font=small,labelfont=bf]{caption}
\usepackage[a4paper, left=2.5cm, right=1.5cm, top=2.5cm, bottom=2.5cm]{geometry}

\graphicspath{ {./images/} }
\setlength{\droptitle}{-3cm}
\hypersetup{ colorlinks=true, linktoc=all, linkcolor=blue }
\pagenumbering{arabic}

\begin{document}
    \subsection{Схема Горнера}
    
    \(f(x)\) на \((x-c)\), \(f(x) = a_nx^n+a_{n-1}x^{n-1}+...+a_2x^2+a_1x+a_0\)

    \(f(x) = (x-c)\cdot \varphi(x) + r = (x-c)(b_{n-1}x^{n-1}+...+b_1x+b_0) + r = b_{n-1}x^n+b_{n-1}x^{n-1}+...+b_1x^2+b_0x\)
    \(- cb_{n-1}x^{n-1} - cb_{n-2}x^{n-2} - ... - cb_1 x - cb_0 + r\)

    \(\begin{cases}
        a_n = b_{n-1}\\
        a_{n-1} = b_{n-2} - cb_{n-1}\\
        a_{n-2} = b_{n-3} - cb_{n-2}\\
        \vdots\\
        a_{1} = b_{0} - cb_{1}\\
        a_{0} = r - cb_{0}\\
    \end{cases}\)
    
    
    \(\begin{cases}
        b_{n-1} = a_n\\
        b_{n-2} = a_{n-1} + cb_{n-1}\\
        b_{n-3} = a_{n-2} + cb_{n-2}\\
        \vdots\\
        b_{0} = a_{1} + cb_{1}\\
        r = a_{0} + cb_{0}\\
    \end{cases}\)
    
    \begin{tabular}{c | c | c | c | c | c |}
        & \(a_n\) & \(a_{n-1}\) & ... & \(a_1\) & \(a_0\)\\
        \hline
        & \(+0\) &\(+c\cdot b_{n-1}\) & ... & \(+c\cdot b_1\) & \\
        \hline
        \(c\) & \(b_{n-1}\) &\(b_{n-2}\) & ... & \(b_0\) & \(r\)\\
    \end{tabular}


    \textbf{Пример.}
    \(2x^5 - x^4 - 3x^3 + x - 3 \) на \(x - 3\)

    \begin{tabular}{c | c | c | c | c | c | c |}
        & \(2\) & \(-1\) & \(-3\) & \(0\) & \(1\) & \(-3\)\\
        \hline
        \(3\) & \(2\) &\(5\) & \(12\) & \(36\) & \(109\) & \(324\)\\
    \end{tabular}

    Итог:
    \(2x^5 - x^4 - 3x^3 + x - 3 = (2x^4 + 5x^3 + 12x^2 + 36x + 109) \cdot (x - 3) + 324\)
    

    
    \subsection{Кратные корни множеств и корни производной от многочелнов.}

    \( x^2 + 2x + 1 = (x+1)^2 \)

    \textbf{Определение.} Корень \(x=c\) многочлена \(f(x)\) называется корнем кратности \(k\), если \(f(x) \vdots (x-c)^k\) и \(f(x) \not \vdots (x-c)^{k+1}\)

    \( f(x) = a_{n}x^{n} + a_{n-1}x^{n-1} +...+ a_{2}x^{2} + a_{1}x + a_{0} \)

    \( a_j; x \in \mathbb{C}\) %%todo
    
    \( f'(x) = na_{n}x^{n-1} + (n-1)a_{n-1}x^{n-2} + ... + a_1 \)

    \( f^{n}(x) \not \equiv 0 = a_{n}n!\)
    \( f^{n + 1}(x) \equiv  0 \)
    

    \textbf{Теорема.} \(k\)-кратный корень многочлена \(f(x)\) является \(k-1\)-кратным корнем производной \(f'(x)\).
    
    \(\uparrow\) Пусть \(x = c\) --- \(k\)-кратный корень \(f(x) \Rightarrow f(x) = (x-c)^k\cdot \varphi(x), \varphi(x)\not \vdots (x-c)\)
   
    \(f'(x) = k(x-c)^{k-1}\cdot \varphi(x) + (x-c)^k\varphi'(x) =\)
    \( (x-c)^{k-1}[k\varphi(x) + (x-c)\varphi'(x)]\)
    
    \(\begin{array}{l}
        k\varphi(x) \not \vdots (x-c)\\
        (x-c)\varphi'(x) \vdots (x-c)
    \end{array} \Rightarrow [k\varphi(x) + (x-c)\varphi'(x)] \not \vdots (x-c) \)
   
    \(\Rightarrow (x-c)\) --- корень кратности \(k-1\) для \(f'(x)\ \downarrow\)
    
    \textbf{Замечание.} Если \(x = c\) --- это корень кратности \(k\), то для производной \(m_{(m \leq k)}\)-го порядка \(x = c\) является корнем кратности \(k-m\), и впервые не будет корнем \(k\)-го порядка от \(f(x)\).
    
    

    
    \subsection{Основная теорема алгербры и её следствия}
    
    \textbf{Теорема.} Любой многочлен \(f(x)\) над \(\mathbb{C}\), имеет хотя бы один корень. \\

    В 1799г: в \(\mathbb(R)\)
    
    \(n\) --- неч

    \(f(x) a_n > 0\)\\
    \(\lim_{x \to +\infty} f(x) = +\infty\)\\
    \(\lim_{x \to -\infty} f(x) = -\infty\)

    \(f(x) a_n < 0\)\\
    \(\lim_{x \to +\infty} f(x) = -\infty\)\\
    \(\lim_{x \to -\infty} f(x) = +\infty\)

    \(n\) --- чётн?

    \textbf{Следствие.} В \(\mathbb{C}\) любой многочлен \(f(x)\) можно представить в виде: \( a_{n}x^{n} + a_{n-1}x^{n-1} + ... + a_1x + a_0 = f(x) = a_{n}(x - c_1)(x - c_2)...(x - c_n)\)
    \(\exists!\) с точностью до перестановки множителей.

    \(\uparrow \exists\) по ОТА для \(f(x)\ \exists\) хотя бы один корень, пусть это \(x = c_1\), тогда по следствию из т. Безу
    \begin{enumerate}
        \item \(f(x) = (x-c_1)\cdot \varphi(x)\)
        \item \(\varphi(x)\) по ОТА \(\exists\) хотя бы один корень, пусть это \(x = c_2 \Rightarrow\)
        
        \(\Rightarrow f(x) = (x-c_1)\varphi(x) = (x-c_1)(x-c_2)\psi(x)\) и т.д.
        \item[\(n\).] \(f(x) = a_n(x-c_1)(x-c_2)...(x-c_n)\)
    \end{enumerate}
    

    \(\uparrow \nexists\) пусть два представления
    \( f(x) = a_n(x-c_1)(x-c_2)...(x-c_n) \)
    \( f(x) = a_n(x-b_1)(x-b_2)...(x-b_n) \)
    
    \underline{без кратности}
    
    \(\forall c_i\  \exists\ b_j = b_i = c_i ?\)
    
    \( (x-c_1)(x-c_2)...(x-c_n) = (x-b_1)(x-b_2)...(x-b_n) \) (*)
    
    \underline{От противного.} Пусть для некоторого \(x = c_i\ \exists\) ни одного \(b_j\ :\ b_j = c_i\), тогда в равенстве (*)

    при \(x = c_i\)
    
    \(\begin{array}{l}
        \textrm{левая часть} = 0\\
        \textrm{правая часть} \neq 0
    \end{array}\) невозможно

    \( \Rightarrow \forall c_j\ \exists\ b_j: c_i = b_j\)

    \underline{про кратность:} Пусть \(x = c_i\) --- \(S\) кратный для \(f(x)\) a \(x = b_j = c_i\) --- \(t\) кратный для \( (t \not = S)\) 
    
    \textbf{Замечание.} Покажем, что можем сокращать.

    Пусть \[f(x) = \varphi(x)\psi(x) = \varphi(x)g(x) \Rightarrow \psi(x) = g(x)\]
    \[\Downarrow\]
    \[ \varphi(x)[\psi(x) - g(x)] \equiv 0 \]
    \[\Downarrow\]
    \[\psi(x) \equiv g(x)\]
    
    \( (x - c_i)(..S..) = (x - c_i(..t..))\)
    если \(S < t\) то сокращается на \((x-c_j)^S\)

    \[\Downarrow\]

    (многочлен) \( = (x - c_i)^{t - S}(\)многочлен\()\), при \(t - s \neq 0\)\\
    (многочлен) \(\not \vdots (x-c_i)\)\\
    противоречие на \(x = c_i\) с тем, что \(S \neq t\ \downarrow\)

    
    \textbf{Следствие.} \(c_1, c_2,..., c_n\) --- и только они --- корни.
    
    \textbf{Следствие.} \(f(x) = a_n(x-c_1)^{k_1}(x-c_2)^{k_2}...(x-c_m)^{k_m}\)
    
    \(k_1 + k_2 + ... + k_m = n\)
    
    \textbf{Утверждение.} \(f(x) \equiv g(x) \Leftrightarrow \forall c \in \mathbb{C}\ f(c) = g(c)\)
    
    \textbf{Следствие.} \(f(x);\ g(x) \vdots\)
    
    \(deg\ f(x) \leq n\)\\
    \(deg\ g(x) \leq n\)
    
    и \( \forall c_i \ i=1, n+1\ f(c_i) = g(c_i) \Rightarrow f(x) \equiv g(x) \)
    
    
    \(\uparrow\ \) Если \(f(x) - g(x) \not \equiv 0\)
    \(f(x) - g(x) = h(x) \overset{\textrm{по следствию ОТА}}= \tilde{a}(x) (x-c_1)(x-c_2)...(x-c_{n+1})\)
    
    \(deg\ (f(x)-g(x)) \leq n\)\\
    \(deg\ h(x) \geq n+1\)

    невозможно \(\Rightarrow f(x) \equiv g(x)\downarrow\)
    
    \textbf{Пример.}

    \(\frac{(x-a)(x-b)}{(c-a)(c-b)}+\frac{(x-c)(x-b)}{(a-c)(a-b)} + \frac{(x-a)(x-c)}{(b-a)(b-c)} = 1\)
    
    Имеем квадратное уравнение:
    
    \(x = a\) --- корень\\
    \(x = b\) --- корень\\
    \(x = c\) --- корень
    
    Слева стоит многочлен, \(deg \leq 2\), справа \(deg = 0\), уравнение не обязано быть квадратным.
    
    \subsection{На практике}
        \begin {enumerate}
            \item[\(0\).] Уменьшить степень
            \item \(n\) --- нечётный, \(a_i \in \mathbb(R)\) действительные всегда есть!\\
            целые?
            
                \( a_{n}x^{n} + a_{n-1}x^{n-1} + a_{n-2}x^{n-2} +... + a_1x + a_0 = 0\)\\
                Целые корни --- делители свободного члена.\\

            \(x = x^*\) --- корень
            
            \( \underset{\vdots x_*}{\underbrace{a_{n}x^{n}_* + a_{n-1}x^{n-1}_* + ... + a_1x_*}} = -a_0 \Rightarrow\)
        
            \item Рациональные корни?
            
            \(r = frac{p}{q}\) --- несократимая дробь\\
            \( a_{n}frac{p^{n}}{q^n} + a_{n-1}frac{p^{n-1}}{q^{n-1}} + ... + a_{1}frac{p}{q} = 0\)
            
            домножим на \(q^n\)

            \(a_{n}p^{n} + a_{n-1}p^{n-1}q + ... + a_{1}pq^{n-1} = -a_{0}q^n\)\\
            м.ч. \(\vdots p \Rightarrow a_0 \vdots p\)
            
            \(a_np^n=(-a_{n-1}p^{n-1}q -...-a_0q^n) \vdots q \Rightarrow a_n\vdots q\)

            \item Если  \(z\) --- корень \(f(x)\), то \(\overline{z}\) --- тоже корень, если \(f(x)\) --- многочлен с действительными коэффициентами.
            
            \(\overline{z_1+z_2} = \overline{z_1}+\overline{z_2}\)

            \(\overline{z_1z_2}=\overline{z_1}\cdot\overline{z_2}\)

            \(a_n\overline{z}^n + a_{n-1}\overline{z}^{n-1}+...+a_1\overline{z}+a_0=0\)
        
            \item Если \(x = a + b^{0.5} \Rightarrow x_2 = a - \sqrt{b}\)тоже корень для \(f(x)\)\\ 
        \end{enumerate}

\end{document}