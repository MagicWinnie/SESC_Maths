\documentclass{article}

\usepackage[utf8x]{inputenc}
\usepackage[english,russian]{babel}
\usepackage{cmap}
\usepackage{commath}
\usepackage{amsmath}
\usepackage{amsfonts}
\usepackage{mathtools}
\usepackage{amssymb}
\usepackage{parskip}
\usepackage{titling}
\usepackage{color}
\usepackage{hyperref}
\usepackage{cancel}
\usepackage{enumerate}
\usepackage{multicol}
\usepackage{graphicx}
\usepackage{docmute}
\usepackage[font=small,labelfont=bf]{caption}
\usepackage[a4paper, left=2.5cm, right=1.5cm, top=2.5cm, bottom=2.5cm]{geometry}

\graphicspath{ {./images/} }
\setlength{\droptitle}{-3cm}
\hypersetup{ colorlinks=true, linktoc=all, linkcolor=blue }
\pagenumbering{arabic}

% May 3, 2022

\begin{document}
    \subsection{Фомула Бернули}
    Вывдеем формулу Бернули, позволяющую вычеслить вероятность того, что в серии из $n$ независимых испытаний событий $A$, имеющее вероятность $p$, встретится $m$ раз.
    Результат серии из $n$ испытаний ——— последовательность из $A$ и $\lnotA$ длины $n$.
    Например, \(A\overline{A}\overline{A}\overline{A}A\)
    Независимость испытаний позволяет вычислить вероятность пояления конкретной серии из $n$ испытаний, в которой событие $A$ произошло  $m$ раз: \(p^mq^{n-m}\), где \(q = 1 - p\).
    

    \textbf{Теорема 1.} Пусть вероятность события $A$ равна $p$, и пусть $P_{mn}$ - вероятность того, 
    что в серии из $n$ независимых испытаний это событие произойдёт $m$ раз. Тогда справедлива формула Бурнулли\\
    \(P_{mn} = C_n^m \cdot p^m \cdot q^{n-m}\)
    
    
    \(\uparrow\) Благоприятными сериями испытаний являются в данном случае т.е. серии из $n$ испытаний, в которых событие $A$ произошло $m$ раз.
    Каждая такая серия задаётся последовательностью из $m$ букв $A$ и $n-m$ букв $\lnot A$.
    Поэтому общее число таких серий равно числу перестановок с повторениями из $n$ элементов двух типов \\    
    \(P(m, n-m) = \frac{n!}{m! \cdot (n - m)!} = C_n^m\)
    
    %доделать слайд 1 ⬆️

    \textbf{Пример 1.} Какова вероятность, что при 10 подбрасываниях монеты герб выпадет 5 раз? А рпи 20 подбрасываниях 10 раз выпадет герб?
    
    Вероятность выпаддения герба при подбрасывании монеты равна \(\frac{1}{2}\)
    
    По формуле имеем: \( P_{5, 10} = C^5_{10}(\frac{1}{2})^5(\frac{1}{2})^5 = \frac{252}{1024} \approx 0,2461\)
    
    Для 20 подбрасываний: \( P_{10, 20} = C^{10}_{20}(\frac{1}{2})^10(\frac{1}{2})^10 = \frac{184756}{1048576} \approx 0,18\)


    \textbf{Пример 2.} Найти наиболее вероятное число выпадений герба при 10 подбрасываниях монеты.\\
    Надо сравнить $P_{m,10}$ и $P_{m+1, 10}$.\\
    Найдём их отношение и сравним его с единицей:\\
    \(\frac{P_{m, 10}}{P_{m + 1, 10}} = \frac{C_{10}^m}{C_{10}^{m + 1}} = \frac{10! \cdot (m + 1)! \cdot (10 - m - 1)!}{m! \cdot (10 - m)! \cdot 10!} = \frac{m + 1}{10 - m} < 1\)\\
    Отсюда \( m \leq 4 \), т.е.: \( P_{0, 10} < P_{1, 10} < ... P_{4, 10} < P_{5, 10}\)  \(> P_{6, 10} > ... > P_{9, 10} >  P_{10, 10} > \)
    
    \subsection{Геометрическая вероятность}

    \textbf{Пример 1.} Стержень наудачу разламывают на три части. Какова вероятность, что из получившихся отрезков можно будет получить треугольник?
    
    %Нужно сделать надпись "Модель" по центру
    \(\begin{center} \textbf{Модель} \end{center}\)

    Пусть на отрезок $AB$, бросают на удачу точку. %изображение

    \textbf{Определение} Назовём \textif{вероятностью попадания точки на часть этого отрезка} отношение длины этой части к длине всего отрезка (если часть состоит из несколъких кусков, надо сложить длины этих кусков), т.е. \( \frac{|CD|}{|AB|} \).
    Такое определение естественно, так, чем больше цель, тем вероятнее её поразить. Оказывается, что свойства введенного таким образом понятия вероятности очень похожи на свойства рассмотренные ранее. Справедливы следующие утверждения:
    
    \begin{enumerate}
        \item Для любой части отрезка значения вероятности является неотрицательным числом не превышающим 1. Для самого отрезка значение верояности равно 1.
        \item Если часть $X$ и $Y$ не имеют общих точек (несовместны), то \(P(X \cup Y) = P(X) + P(Y)\). 
    \end{enumerate}\\
    
    \textit{На основе этих двух утверждений для геометрических вероятностей можно определить все те же понятия, 
    что и в случае конечного вероятностного пространства, доказать аналоги формул сложения и умножения вероятностей, формулу Байеса и т.д.}

    %доделать слайд 2 ⬆️
    
    \textbf{Замечание:} Вместо отрезка можно взять некоторую геометрическую фигуру, имеющую конечную площадь, и считать вероятностью попасть в часть этой 
    фигуры отношение площадей указанной части и всей фигуры.

    \textbf{Пример 1.} Стержень наудачу разламывают на три части. Какова вероятность, что из получившихся отрезков можно будет получить треугольник?
    
    Заданный стержень можем рассматривать как отрезок [0,1]. Наудачу брошенные точки имеют координаты --- числа $x$ и $y$.\\
    Поскольку 0<=x<=1 , 0<=y<=1, то пару (x,y) 

    1. Если \textbf{$x \leq y$}, то имеем три отрезка с длинами \(x, y-x, 1-y\). Эти три числа должны удовлетворять неравенству треугольника.
    
    %доделать слайд 3 ⬆️

    3. Если \textbf{$x > y$} то имеем три отрезка с длинами \(y, x-y, 1-x\). Выписываем неравенства треугольника:
    %%


    Это множество также представляет собой треугольник $(S_2)$ на плоскости $XOY$\\
    
    %доделать слайд 4 ⬆️

    
    \textbf{Пример 2. (Задача Бюффона)} На плоскости (бесконечной) проведено семейство параллельных прямых (тоже бесконечно). Рассмотрим между соседними прямыми равно $l$. На эту плоскость бросается наудачу отрезок длины $l$. Какова вероятность, что отрезок пересекается хотя бы с одной из прямых семейства?   
    
    Обозначим через $y$ --- растояние от верхнего конца отрезка до ближайшей снизу прямой.
    Проведём луч с началом в верхней левой точке отрезка параллельно прямым семейства и идущих направо. обозначим через $x$ --- угол между лучём и отрезком.
    Мы получим пару чисел $(x, y)$, удовлетворяющую неравенствам \(0 \leq x < \pi\), \(0 \leq y \leq 1\)
    
    %схема%
    
    Точка $(x, y)$ с такими координатами наудачу брошена в прямоугольник. Для того чтобы отрезок пересекался хотя бы с одной из прямых семейства необходимо и достаточно, чтобы \(y' = l\sin x \geq y\)
    \( S = \int^{\pi}_0l\sin xdx = l(-\cos x)|^{\pi}_0 = -l(-1-1) = 2l\)

    %график%
    
    Так как площадь прямоугольника равна $\pi l$, то \(P = \frac{2l}{\pi l} = \frac{2}{\pi} \approx 0.6366\)

    %Доделать слайд 5 ⬆️
    
    \end{document}