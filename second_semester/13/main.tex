\documentclass{article}

\usepackage[T2A]{fontenc}
\usepackage[utf8]{inputenc}
\usepackage[russian]{babel}
\usepackage{commath}
\usepackage{amsmath}
\usepackage{parskip}
\usepackage{color}
\usepackage{hyperref}
\usepackage[a4paper, left=2.5cm, right=1.5cm, top=2.5cm, bottom=2.5cm]{geometry}

\hypersetup{
    colorlinks=true, %set true if you want colored links
    linktoc=all,     %set to all if you want both sections and subsections linked
    linkcolor=blue,  %choose some color if you want links to stand out
}

\pagenumbering{arabic}

\begin{document}
  % \tableofcontents
  % \thispagestyle{empty}
  % \setcounter{tocdepth}{5}
  % \newpage

  \addtocontents{toc}{\protect\contentsline{section}{\protect\numberline{}Элементы теории чисел. Теория сравнений.}{}{}}
  \title{Элементы теории чисел. Теория сравнений.}
  \author{Ученик 10-4 класса Оконешников Д.Д. по лекции Протопоповой Т.В.}
  \date{14 апреля 2021 г. по лекции от 12 января 2021 г.}
  \maketitle

  \section{Лекция №12}
  
  
\end{document}
