\documentclass{article}
% PKGS START
\usepackage[utf8x]{inputenc}
\usepackage[english,russian]{babel}
\usepackage{cmap}
\usepackage{commath}
\usepackage{amsmath}
\usepackage{amsfonts}
\usepackage{mathtools}
\usepackage{amssymb} 
\usepackage{parskip}
\usepackage{titling}
\usepackage{color}
\usepackage{hyperref}
\usepackage{cancel}
\usepackage{enumerate}
\usepackage{graphicx}
\usepackage[a4paper, left=2.5cm, right=1.5cm, top=2.5cm, bottom=2.5cm]{geometry}
% PKGS END
% INIT START
\graphicspath{ {./images/} }
\setlength{\droptitle}{-3cm}
\hypersetup{
    colorlinks=true, %set true if you want colored links
    linktoc=all,     %set to all if you want both sections and subsections linked
    linkcolor=blue,  %choose some color if you want links to stand out
}

\pagenumbering{arabic}
% INIT END
\begin{document}
    \section{Элементы логики}

    \subsection{Взаимно обратные и взаимно противоположные теоремы, необходимые и достаточные условия}

    Большинство теорем школьного курса имеют такой вид:
    \[\forall x \in U\ A(x) \Rightarrow B(x)\]
    \textbf{Теорема 1.} \(\forall n \in \mathbb{N}\) если \(A(n)=\) \{сумма цифр числа \(n \vdots 3\)\}, то \(B(n)=\) \{число \(n \vdots 3\)\}

    \textbf{Теорема 2.} \(\forall a, b \in \mathbb{Z}\), если \(A(a,b)=\ \{a \vdots 7\) или \(b \vdots 7\}\), то \(B(a,b)=\{ab \vdots 7\}\)

    \textbf{Теорема 3.} Диагонали ромба взаимоперпендикулярны

    \textbf{Теорема 3\(^\prime\).} \(\forall p \in P\)(\(P\) --- множество всех паралеллограммов), если \(A(p)=\) \{паралелограмм \(p\) является ромбом\}, то \(B(p)=\) \{диагонали паралелограмма \(p\) взаимно перепендикулярны\}

    \textbf{Теорема 4.} \(\forall q \in \mathbb{Q}\)(\(\mathbb{Q}\) --- множество всех четырёхугольников), если \(A(q)=\) \{четырёхугольник \(q\) является ромбом\}, то \(B(q)=\) \{диагонали четырёхугольника \(q\) взаимно перепендикулярны\}

    \textbf{Определение.} Теоремы \(\forall x \in U\ A(x) \Rightarrow B(x)\) и \(\forall x \in U\ B(x) \Rightarrow A(x)\) называются взаимно \textit{обратными}

    \textbf{Определение.} Если теорема \(\forall x U\ A(x) \Rightarrow B(x)\) верна, то предложение \(A(x)\) называется \textit{достаточным} условием для \(B(x)\), а \(B(x)\) --- \textit{необходимым} условием для \(A(x)\)

    \textbf{Определение.} Теоремы \(\forall x \in U\ A(x) \Rightarrow B(x)\) и \(\forall x \in U\ \overline{A(x)} \Rightarrow \overline{B(x)}\) называются \textit{противоположными}

    Всякая теорема \(\forall x \in U\ A(x) \Rightarrow B(x)\) порождает три ещё теоремы:
    \begin{enumerate}
        \item Обратную \(\forall x \in U\ B(x) \Rightarrow A(x)\)
        \item Противоположную \(\forall x \in U\ \overline{A(x)} \Rightarrow \overline{B(x)}\)
        \item Противоположную обратную \(\forall x \in U\ \overline{B(x)} \Rightarrow \overline{A(x)}\)
    \end{enumerate}

    \textbf{Пример.} Если \(xy\) --- нечётно, то и \(x\) и \(y\) --- нечётны.

    \(\uparrow\) От противного. Пусть \(x\) --- чётное или \(y\) --- чётное. Например, \(x = 2k\), тогда \(xy = 2ky\) --- чётно. Противоречие \(\downarrow\)

    \section{Натуральные числа}

    \subsection{Принцип математической индукции}

    Существуют два типа построения математической теории:
    \begin{enumerate}
        \item Дедуктивный --- утверждения выводятся из нескольких аксиом с помощью логического вывода(дедукции)
        \item Индуктивный --- вывод получается путём заключения от частного к общему(например, на основе опытов)
    \end{enumerate}

    Если можно проверить истинность утверждения \(A(n)\) для всех \(n\), то такой метод называется \textit{полной индукцией}(полный перебор).

    \subsubsection{Метод математической индукции}

    \begin{enumerate}
        \item Пусть \(A(n)\) истинно при \(n=1\)(\textit{база индукции})
        \item И из предположения о том, что \(A(n)\) истинно при \(n=k\)(\(k\) любое \(\in \mathbb{N}\))(\textit{индукционное предположение}) \(\Rightarrow\) что оно истинно при \(n=k+1\)(\textit{индукционный переход}), тогда \(A(n)\) истинно \(\forall n \in \mathbb{N}\)
    \end{enumerate}

    Метод математической индукции можно рассматривать как дедукцию из \textit{аксиом натуральных чисел Пеано}

    \subsubsection{Аксиомы Пеано}

    \begin{enumerate}
        \item Единица есть натуральное число, которое не следует непосредственно ни за каким другим натуральным числом
        \item Каково бы ни было натуральное число \(n,\ \exists!\) натуральное число \(n^\prime\), которое непосредственно следует за \(n\)
        \item Каждое \(n \in \mathbb{N}\), отличное от 1, следует непосредственно лишь за одним натуральным числом
        \item Если некоторое множество натуральных чисел \(M\) содержит число 1 и вместе с каждым \(n \in M\) содержит непосредственно следующее за ним число \(n^\prime\), то \(M = \mathbb{N}\)(\textit{аксиома математической индукции})
    \end{enumerate}

    \textbf{Пример.} С помощью метода математической индукции доказать истинность предположения \(\forall n \in \mathbb{N}\ A(n) = \{(5*2^{3n-2} + 3^{3n-1}) \vdots 19\}\)
    \begin{enumerate}
        \item \(n=1\) \quad \(5*2^{3-2} + 3^{3-1} = 5*2 + 9 = 19\ \vdots 19\) --- верно
        \item \(n=k\) \quad Пусть \((5*2^{3k-2} + 3^{3k-1}) \vdots 19\)
        \item Докажем, что при \(n = k+1\) утверждение тоже верно: \((5 * 2^{3(k+1)-2} + 3^{3(k+1)-1}) \vdots 19\)
        
        \(5 * 2^{3(k+1)-2} + 3^{3(k+1)-1} = 5 * 2^{3k+3-2} + 3^{3k+3-1} = 5 * 8 * 2^{3k-2} + 27 * 3^{3k-1} = (8 \underset{\vdots 19}{\underbrace{(5*2^{3k-2} + 3^{3k-1})}} + \underset{\vdots 19}{\underbrace{19 * 3^{3k-1}}})\ \vdots 19\), ч.т.д.
    \end{enumerate}
\end{document}