\documentclass{article}

\usepackage[T2A]{fontenc}
\usepackage[utf8]{inputenc}
\usepackage[russian]{babel}
\usepackage{commath}
\usepackage{amsmath}
\usepackage{amsfonts}
\usepackage{mathtools}
\usepackage{amssymb} 
\usepackage{parskip}
\usepackage{titling}
\usepackage{color}
\usepackage{hyperref}
\usepackage{cancel}
\usepackage{enumerate}
\usepackage{graphicx}
\usepackage[a4paper, left=2.5cm, right=1.5cm, top=2.5cm, bottom=2.5cm]{geometry}

\graphicspath{ {./images/} }
\setlength{\droptitle}{-3cm}
\hypersetup{
    colorlinks=true, %set true if you want colored links
    linktoc=all,     %set to all if you want both sections and subsections linked
    linkcolor=blue,  %choose some color if you want links to stand out
}

\pagenumbering{arabic}

\begin{document}
    \begin{titlepage}
        \begin{center}
            \vspace*{1cm}
                
            \Huge
            \textbf{Математика 1-й семестр, 10-й класс}
                
            \vspace{5cm}
            
            \Large
            \textbf{Ученики 10-4 класса}\\
            \textbf{Оконешников Д.Д., Паньков М.А. и Кангелдиева А.С.}\\
            \textbf{по лекциям к.ф.-м.н. Протопоповой Т.В.}

            \vfill

            \vspace{0.8cm}

            \Large
            
            Для внутреннего использования
            
            Россия, г. Новосибирск, СУНЦ НГУ, 2021 год
            
            v1.0.0
                
        \end{center}
    \end{titlepage}
    \newpage

    \tableofcontents
	\thispagestyle{empty}
	\setcounter{tocdepth}{5}
	\newpage
    
    \addtocontents{toc}{\protect\contentsline{section}{\protect\numberline{}Первый семестр}{}{}}
    \section{Понятие множества. Подмножества (включения).}
\end{document}